\chapter*{Considera��es pessoais}

Apesar de ter sido extremamente estressante lidar com o desenvolvimento desse trabalho, a gratifica��o foi muito maior. O tema foi sugerido pelo professor Alfredo, apesar de, na �poca, eu j� ter definido um tema, e n�o me arrependo de ter seguido com o tema desenvolvido aqui. Revi notas de aulas de aulas das quais gostei muito de ter na gradua��o, como Estrutura de Dados, com o professor Paulo Feofiloff, e Algoritmos em Grafos, com o professor Marcel Kenji. Essas revis�es me lembraram do porqu� eu gosto do que fa�o e porqu� quero continuar nessa �rea e, possivelmente, continuar pesquisando sobre o uso de FPGAs por desenvolvedores de \textit{software}.

O principal motivo para ter seguido a sugest�o de tema do professor Algredo foi a mem�ria do est�gio que fiz no exterior, onde meus colegas lidaram com a elabora��o dos circuitos de novos prot�tipos de placas e eu, sem poder ajudar, me frustrava bastante. Talvez disseminar mais este tema possa ajudar a desenvolvedores de \textit{software} a n�o "sofrer" tanto quando quiserem aprender sobre o desenvolvimento de \textit{hardware}.

O fato de estagiar e ainda cursar algumas mat�rias da gradua��o durante o per�odo de desenvolvimento do trabalho tamb�m atrapalhou bastante a pesquisa, apesar dos efeitos ben�ficos dessas atividades em outros �mbitos pessoais.

O resultado final das experi�ncias foi frustrante, devido � quebra da expectativa em se chegar em um resultado palp�vel sobre a ferramenta e os algoritmos implementados. E, mesmo assim, considerando o quanto tive que aprender e pesquisar desde os conceitos mais b�sicos dos assuntos fundamentais para a elabora��o deste trabalho (como na estrutura de FPGAs e s�ntese de alto n�vel), eu posso dizer que foi bom ver o resultado das pesquisas e implementa��es feitas, al�m dos conhecimentos obtidos. 