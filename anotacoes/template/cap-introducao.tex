%% ------------------------------------------------------------------------- %%
\chapter{Introdu��o}
\label{cap:introducao}

A comunidade de desenvolvedores de hardware exibe, hoje, um grande interesse e necessidade de ferramentas que facilitem a implementa��o de algoritmos e fun��es em hardware. Nessa situa��o, junto da evolu��o de tecnologias como as placas FPGAs, temos o progresso da s�ntese de alto n�vel, que permite a tradu��o de algoritmos para circuitos l�gicos. Logo, seria interessante um estudo sobre a s�ntese de alto n�vel voltada para FPGAs, implementando algoritmos tanto para FPGAs, utilizando ferramentas de s�ntese de alto n�vel como os arcabou�os OpenCL e LegUp, quanto para computadores multicore. Al�m disso, � importante avaliar o qu�o vantajoso seria implementar tais algoritmos em FPGAs em termos de desempenho, fazendo-se necess�rio, ent�o, o benchmarking e posterior avalia��o de desempenho dos algoritmos implementados nas diferentes arquiteturas.

\begin{small}
\begin{verbatim}
Modos de cita��o:
indesej�vel: [AF83] introduziu o algoritmo �timo.
indesej�vel: (Andrew e Foster, 1983) introduziram o algoritmo �timo.
certo : Andrew e Foster introduziram o algoritmo �timo [AF83].
certo : Andrew e Foster introduziram o algoritmo �timo (Andrew e Foster, 1983).
certo : Andrew e Foster (1983) introduziram o algoritmo �timo.
\end{verbatim}
\end{small}

Uma pr�tica recomend�vel na escrita de textos � descrever as legendas das
figuras e tabelas em forma auto-contida: as legendas devem ser razoavelmente
completas, de modo que o leitor possa entender a figura sem ler o texto onde a
figura ou tabela � citada.  

Apresentar os resultados de forma simples, clara e completa � uma tarefa que
requer inspira��o. Nesse sentido, o livro de ,
\emph{The Visual Display of Quantitative Information}, serve de ajuda na
cria��o de figuras que permitam entender e interpretar dados/resultados de forma
eficiente.

