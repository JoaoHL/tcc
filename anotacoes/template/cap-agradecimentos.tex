\chapter*{Agradecimentos}

Se eu realmente fosse agradecer todas as pessoas que eu sinto que tiveram um impacto positivo real na confec��o deste trabalho, esse provavelmente seria o maior cap�tulo dele.

Obviamente, o primeiro agradecimento � para meus pais, que aguentaram todos os bons e maus momentos do trabalho, da gradua��o e da vida, ouviram minhas ideias para resolver os problemas insurgentes mesmo sem entender o que estava falando, sempre me apoiaram nas minhas decis�es benignas e me advertiram sobre os impactos das m�s decis�es, tamb�m.

O segundo agradecimento vai para a pessoa que me ajudou a me motivar no meu pior momento da gradua��o, meu amigo germano-brasileiro Mauro Zanella. Se ele n�o tivesse me chamado para sair da faculdade por $1$ semestre para estagiar com ele em Friedrichshafen, era capaz de que eu n�o conseguisse continuar a gradua��o e me formar.

N�o menos importante, agrade�o a todos os meus amigos e pessoas que passaram pela minha vida para me trazer ensinamentos importantes sobre ela, al�m de terem me apoiado quando mais precisei, nos picos de estresse e ansiedade. Passando pela Santa Batcaverna, pelas Mi�angueiras, at� chegar no CC e no Simpl�o, todos tiveram um papel importante, mesmo os que me causaram mal em algum momento. N�o citarei nomes porque s� a ideia de esquecer algu�m j� me incomoda bastante, mas quem me ajudou sabe o que fez, o que tamb�m torna desnecess�ria as cita��es espec�ficas.

Por fim, agrade�o tamb�m ao meu orientador Alfredo Goldman, que entre brincadeiras e conv�vios sociais (tamb�m conhecido como \textit{bullying} para os menores de $18$ anos), me aconselhou e ajudou muito durante a maior parte da gradua��o. Devo boa parte do meu bom-senso profissional e acad�mico a ele.

N�o fossem por todas essas pessoas, eu n�o seria quem eu sou, nem estaria onde estou, ou saberia o que sei. Agrade�o a todos, ainda mais o que se mantiveram por perto nas piores fases.