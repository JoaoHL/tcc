%% ------------------------------------------------------------------------- %%
\chapter{Conclus�es}
\label{cap:conclusoes}

Os resultados dos experimentos mostram que o uso de uma placa FPGA dedicada � execu��o de um algoritmo pode trazer um desempenho igual ou melhor para entradas suficientemente grandes. No entanto, uma diferen�a grande entre a frequ�ncia de \textit{clock} de dispositivos ASIC e FPGA pode se tornar um empecilho no uso exclusivo de FPGAs. Por isso nota-se uma tend�ncia ao uso h�brido de FPGAs com outros dispositivos, como GPUs e CPUs, enquanto procura-se aumentar a capacidade de processamento dos \textit{chips} reprogram�veis.

Um caso not�vel sobre a integra��o desses dispositivos � o da Intel, que planeja integrar circuitos reprogram�veis em seus processadores\footnote{\url{https://www.networkworld.com/article/3230929/data-center/intel-unveils-hybrid-cpu-fpga-plans.html}}, a fim de possibilitar o aumento de desempenho com a personaliza��o do circuito de acordo com as necessidades do usu�rio, dispobilizando APIs e tutoriais de linguagens para s�ntese de alto n�vel (e.g OpenCL).

Quanto ao LegUp, h� a desvantagem de usar em seu fluxo de execu��o, ferramentas propriet�rias cujas vers�es gratuitas podem restringir seu funcionamento integral. Um exemplo disso s�o as mensagens de aviso do simulador de \textit{hardware} do arcabou�o, que usa o ModelSim. As mensagens avisam que um n�mero muito grande de instru��es no arquivo Verilog pode afetar de forma adversa o desempenho e qualidade da simula��o, o que pode ser extremamente prejudicial para pesquisas futuras. Ainda assim, em termos de usabilidade e complexidade, o arcabou�o se mostrou uma �tima ferramenta para fins acad�micos.

O conhecimento obtido e apresentado sobre o processo de s�ntese de alto n�vel e poss�veis ferramentas que a utilizem com placas-alvo de maior acessibilidade, como o LegUp em rela��es �s placas Cyclone IV\footnote{\url{https://www.intel.com/content/www/us/en/products/programmable/fpga/cyclone-iv.html}} da Altera, ajuda a difundir a cultura de desenvolvimento em FPGAs, atraindo jovens pesquisadores � �rea de pesquisa em \textit{hardware}. Dada a disparidade entre o n�mero de pesquisadores e desenvolvedores nas �reas de \textit{software} e \textit{hardware}, fazer uma ponte entre elas pode ajudar a acelerar a cria��o de novas tecnologias.

