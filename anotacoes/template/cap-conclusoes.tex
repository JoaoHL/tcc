%% ------------------------------------------------------------------------- %%
\chapter{Conclus�es}
\label{cap:conclusoes}

Os resultados dos experimentos foram inconclusivos em termos de compara��o de desempenho entre algoritmos implementados em circuitos eletr�nicos e algoritmos em \textit{software}. Por consequ�ncia, no que se diz respeito � avalia��o de desempenho, pouco foi esclarecido para os algoritmos descritos neste trabalho.

Mesmo sendo uma ferramenta eficaz, gratuita e de c�digo aberto, o LegUp tem a desvantagem de usar, em seu fluxo de execu��o, ferramentas propriet�rias cujas vers�es gratuitas podem restringir seu funcionamento integral. Um exemplo disso s�o as mensagens de aviso do simulador de \textit{hardware} do arcabou�o, que usa o ModelSim. As mensagens avisam que um n�mero muito grande de instru��es no arquivo Verilog pode afetar de forma adversa o desempenho e qualidade da simula��o, o que pode ser extremamente prejudicial para pesquisas futuras.

Apesar disso, o conhecimento obtido sobre o processo de s�ntese de alto n�vel e poss�veis ferramentas que a utilizem com placas-alvo de maior acessibilidade, como o LegUp em rela��es �s placas Cyclone IV\footnote{\url{https://www.intel.com/content/www/us/en/products/programmable/fpga/cyclone-iv.html}} da Altera, ajuda a difundir a cultura de desenvolvimento em FPGAs, atraindo jovens pesquisadores � �rea de pesquisa em \textit{hardware}. Dada a disparidade entre o n�mero de pesquisadores e desenvolvedores nas �reas de \textit{software} e \textit{hardware}, fazer uma ponte entre elas pode ajudar a acelerar a cria��o de novas tecnologias.

